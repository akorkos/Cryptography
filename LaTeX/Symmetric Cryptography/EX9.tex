\section{Άσκηση 4.7}

\subsection{Εκφώνηση}

Εξετάστε αν ισχύει το avalanche effect στο AES-128. Αναλυτικότερα, φτιάξτε αρκετά ζευγάρια ($\ge 30$) μηνυμάτων $(m_1, m_2)$ που να διαφέρουν σ ένα bit. Εξετάστε σε πόσα bits διαφέρουν τα αντίστοιχα κρυπτομηνύματα. Δοκιμάστε με δύο καταστάσεις λειτουργίας: ECB και CBC (η δεύτερη θέλει και IV block). Τα μήκη των μηνυμάτων που ϑα χρησιμοποιήσετε να έχουν μήκος διπλάσιο του μήκους ενός block. Δηλαδή για τον AES, να είναι μήκους 256-bits.

\subsection{Λύση} 

\subsubsection{Λεπτομέρειες για την υλοποίηση}

Εκτέλεση αρχείου:

\begin{center}
    \t{python avalanche.py}
\end{center}

Δοκιμάσθηκαν 200 τυχαία ζευγάρια δυαδικών μηνυμάτων, χρησιμοποιώντας τυχαία δυαδικών κλειδιά μήκους 128-bit αντίστοιχα και για το IV. Για την χρήση του AES, έγινε χρήση της βιβλιοθήκης \href{https://pypi.org/project/pycryptodome/}{Crypto} όπου εξετάσθηκαν και οι δυο λειτουργίες.

\subsubsection{Συμπέρασμα}

Έπειτα από αρκετές προσομοιώσεις, οι μέσοι όροι διαφορών στα ζεύγη των μηνυμάτων έχουν ως εξής: για \t{ECB} 64-bit και για \t{CBC} 128-bit. Αυτό σημαίνει πως στην λειτουργία \t{ECB} η αλλαγή ενός bit επιφέρει αλλαγή στο $\approx 25\%$ του μηνύματος άρα, δεν παρουσιάζεται το avalanche effect σε αυτή την λειτουργία. Αντιθέτως, στην λειτουργία \t{CBC} αλλάζει το $\approx 50\%$ του μηνύματος που σημαίνει ότι υπάρχει το avalanche effect.



