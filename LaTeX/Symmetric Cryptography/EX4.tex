\section{Άσκηση 2.5}

\subsection{Εκφώνηση}

Να αποδείξετε ότι, αν στο σύστημα μετατόπισης διαλέγουμε τυχαία τα κλειδιά από το σύνολο $\{0, 1, ..., 23\}$, τότε το σύστημα έχει τέλεια ασφάλεια.

\subsection{Λύση} 

\subsubsection{Μαθηματική επίλυση}

Για να έχει ένα σύστημα τέλεια ασφάλεια (Perfect Secrecy) θα πρέπει κατά C. Shannon να ισχύει το εξής: 

\begin{definition}
    Έστω $(\mathcal{M}, \mathcal{C}, \mathcal{K}, E, D)$ έχει τέλεια ασφάλεια, αν $\forall (m_1, m_2) \in \mathcal{M}$ και για $c \in \mathcal{C}$ ισχύει:
    \begin{equation}
        Pr[k \leftarrow \mathcal{K}: E(m_1, k) = c] = Pr[k \leftarrow \mathcal{K}: E(m_2, k) = c], \forall k \stackrel{\$}{\leftarrow} \mathcal{K}
    \end{equation}
\end{definition}

% https://www.ics.uci.edu/%7Estasio/fall04/lect1.pdf
Θα εξετασθεί για αρχή, τα μηνύματα μήκους ενός χαρακτήρα ($l = 1$). 

\begin{proof}
    Για κάθε γραμμα $m$ και $c \in C$ όπου $\mathcal{C} = \mathcal{M} = \{$\text{Α, Β, ..., Ω}$\}$, υπάρχει μοναδικό $k = (c - m \mod 24)$ τ.ω. $E(m, k) = (m + k \mod 24) = c$. Συνεπώς, για κάθε $m, c$ ισχύει $Pr_{k \leftarrow \mathcal{K}}[k \leftarrow \mathcal{K}: E(m, k) = c] = 1/24$, όπου πληρεί τον ορισμό του C. Shannon.
\end{proof}

Στην συνέχεια θα εξετασθεί για μηνύματα μήκους μεγαλύτερο από 1 ($l > 1$). 

\begin{proof}
    Έστω $m_1 = \text{ΑΒ }, m_2 = \text{ΑΩ και } c = \text{ΒΓ}$. Τότε υπάρχει κλειδί $k \in \mathcal{K}$ τ.ω. $E(m_1, k) = c$, για $k = 1$. Ωστόσο, για κάθε $k \in K$ υπάρχει $E(m_2, k) \neq c$ και συνεπώς $Pr_{K \leftarrow \mathcal{K}}[E(m_1, K) = c] = 1/24$, όμως $Pr_{K \leftarrow \mathcal{K}}[E(m_2, K) = c] = 0$ άρα, δεν πληρείτε ο ορισμός C. Shannon.
\end{proof}

\subsubsection{Συμπέρασμα}

Από τα παραπάνω, καταλήγουμε στο ότι ένα σύστημα μετατόπισης μπορεί να έχει τέλεια ασφάλεια αν-ν το μέγεθος το μηνύματος είναι ίσο με 1.