\section{Άσκηση 2.6}

\subsection{Εκφώνηση}

Υλοποιήστε τον OTP αφού αρχικά μετατρέψετε το μήνυμα σας σε bit με χρήση του παρακάτω πίνακα. Θα πρέπει να δουλεύει η κρυπτογράφηση και η αποκρυπτογράφηση. Το μήνυμα δίνεται κανονικά και έσωτερικά μετατρέπεται σε bits. Το κλειδί είναι διαλεγμένο τυχαία και έχει μήκος όσο το μήκος του μηνύματος σας. Το αποτέλεσμα δίνεται όχι σε bits αλλά σε λατινικούς χαρακτήρες.

\subsection{Λύση} 

\subsubsection{Λεπτομέρειες για την υλοποίηση}

Εκτέλεση αρχείου:

\begin{center}
    \t{python otp.py <μήνυμα>}
\end{center}

Τα argument <μήνυμα> είναι προαιρετικό σε περίπτωση που θέλει ο χρήστης να χρησιμοποιήσει το One Time Pad με κάποιο δικό του μήνυμα.

\subsubsection{Συμπέρασμα}

Δίνοντας το μήνυμα:

\begin{center}
    \t{MISTAKES ARE AS SERIOUS AS THE RESULTS THEY CAUSE}
\end{center}

Για κάποια εκτέλεση του κώδικα (καθώς βασίζεται στην ψευδοτυχαιότητα) έχουμε το κρυπτογραφημένο μήνυμα: 

\begin{center}
    \t{TR-!W!(UOYPLS)LRNLZSSF?LLXYY)HFSPTPCGOJVS}
\end{center}

Όπου θα αποκρυπτογραφηθεί ως εξής:

\begin{center}
    \t{MISTAKESAREASSERIOUSASTHERESULTSTHEYCAUSE}
\end{center}